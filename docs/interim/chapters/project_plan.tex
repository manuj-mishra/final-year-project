\chapter{Project Plan}

This project is based on improving and building upon very recent work in a relatively new line of exploration.
As such, much of the work is experimental.
This project demands a level of familiarity with the relevant theory, methods, and technologies around CA and machine learning which is developed in the early stages.
The middle stages begin by reproducing existing results from recent work on CA morphogenesis and regeneration. This includes new research into morphogenesis for graph CA. 
Following this, the key goal is to developing theoretical ideas to improve the methods used to produce these results. These could be more efficient versions of existing ideas, extrapolating ideas from one domain to another (e.g. applying pattern-damage training methods from Mordvintsev et al \cite{mordvintsev2020growing} to the context of graph CA introduced by Grattarola et al. \cite{grattarola2021learning}), or even entirely novel ideas.
Finally, the project aims to test and improve the efficacy of these methods through experimentation and iteration. 
At this stage, extension goals could also be addressed. 
These include practical work like producing rich demonstrations to allow layman audiences to interact with the research. 
They could also include theoretical work like investigating the connection between periodic and fixed-point solutions in CA rule space and devising a way to incentivise a system to converge to a fixed point solution over a periodic one.
Each of these 3 stages is expected to take a roughly equal amount of time.
Throughout each of the stages, the project write-up will continue to develop.
The interim report will be written during the latter half of stage 1 and the earlier half of stage 2.
This will provide a strong basis for the final report which will be written in the latter half of stage 2 and the earlier half of stage 3.

\section{Key Milestones}
The key milestones in this project along with the projected dates of completion are outlined as follows.\\

\textbf{Nov / Dec 2021}: The first milestone is to decide on specific aims for the project. 
This was achieved by the planned deadline. 
The goals for this project came naturally out of an exploration into the literature around morphogenesis in cellular automata.
Since CA are a relatively mature concept, there is a large body of research in this area.
However, the background reading for this project focused on a subset of this body, concerning the applications of machine learning to learn transition rules.
Since the seminal paper by Wulff and Hertz in 1992 \cite{wulff1992learning}, this was a relatively stagnant area of research until the breakthrough work by Mordvintsev et al in 2020 \cite{mordvintsev2020growing}.
This opened up many possible areas of exploration including self-classifying CAs \cite{randazzo2020self-classifying}, adversarial attacks on morphogenetic CAs \cite{randazzo2021adversarial}, and learning graph CA \cite{grattarola2021learning}.\\

\textbf{Jan / Feb 2022}: Following this, the next goal is to write an interim report which introduces the topic, lays out preliminary knowledge, discusses related work, and details a plan for project execution and evaluation.
It also lays a strong basis for the final project report.
This goal was also achieved on time.
The next technical milestone is to replicate the procedure outlined in the Learning Graph Cellular Automata paper \cite{grattarola2021learning} and reproduce the results on the 5 example point clouds.
It also involves experimenting on new point clouds, eliciting both periodic and fixed-point behaviour.
This will allow me to deeply familiarise myself with the methods used in this paper and gain an intuition for ways in which they could be improved.\\

\textbf{Mar / Apr 2022}: The next milestone would be to devise appropriate machine learning models to train a GCA to perform morphogenesis on point clouds that are smaller in number than the target. 
This will involve going beyond current research on GCA to develop a better understanding of shape and structure. 
For example, the GCA will need to understand what it means for two point clouds of differing density to both be in "the shape of a bunny". 
This will require a more complex metric than the simple Euclidean distance between training points and target points that was used when training and testing on point clouds of equal density.
This training method should then be adapted to induce regenerative behaviour in solutions.
One promising avenue of exploration here would be the damage-based training techniques seen in Growing Cellular Automata \cite{mordvintsev2020growing}.\\

\textbf{May / Jun 2022}: At this point, the goals for the final two months of this project are flexible. If the project is progressing slower than planned, then these months provide a good opportunity to catch up. In the event of a pivot, these months also provide some safety buffer to expand the scope of the project in a different direction from a fall-back position. For example, if the damage-based training proves to be inadequate at producing regenerating solutions in the graph context, then exploring new techniques from different areas of research and developing new ideas will require some additional time that these months can provide.
However, if the project is progressing well, then these months will be used to engage with extension goals.
One extension goal is to produce explorable explanations of the research.
This will involve rewriting some of the machine learning models in Tensorflow.js and producing interactive web demonstrations in a style similar to that used by the Distill journal \cite{distill}.
Another extension goal is to explore the theory behind different types of class 2 CAs.
This is of particular interest as it is useful to seperate fixed-point solutions from periodic solutions and understand the difference between them.
The ideal scenario here would be to devise a method of incentivising machine learning systems to converge to fixed-point solutions over periodic solutions.
Although the final report will continue to develop throughout the whole project, a substantial portion of the last month will also be dedicated towards improving and polishing the final report.
    