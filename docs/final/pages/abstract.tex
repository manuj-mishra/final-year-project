
Since the 1970s, John Conway's Game of Life, often shortened to just "Life", captured the imaginations of mathematicians, scientists and philosophers as an example of emergence, self-organization, and complexity arising out of simplicity. Since then, cellular automata (CA) have been used as models of physical\cite{wolf2004lattice}, chemical\cite{gray1983autocatalytic}, biological\cite{deutsch2021bio}, and social phenomena\cite{white2000high}. Recent work has used deep learning to learn CAs that converge to complex states (morphogenesis)\cite{mordvintsev2020growing} or recover a CA's transition function from a set of observations on its simulation (full rule dynamics)\cite{wulff1992learning}.\\

We investigate the use of evolutionary algorithms to this end. We build a cellular automaton simulator and evolutionary algorithm toolkit to address 3 learning objectives. The first is to induce desirable states in life-like CA, the discrete binary-state family of models in which Life belongs. In particular, we use life-like CA to build a procedural maze generator and learn transition functions that produce mazes with desirable qualities like long solution paths and numerous dead ends. The second objective is to learn the full rule dynamics of life-like CA. We show that genetic algorithms can quickly learn the full rule for 100\% of tested life-like CA from less than 10 observations, and usually from a single observation, albeit with longer training time. The third objective is to extend these results to Gray-Scott systems, the family of continuous state automata that simulate the reactions of two diffusive chemicals. These systems are discretizations of Turing's model for morphogenesis, a major theory in developmental biology. We extend the simulator to model continuous automata efficiently and extend the evolutionary algorithm toolkit to implement evolutionary strategies and a wider range of loss functions. In this case, we find that evolutionary algorithms struggle to learn full rule dynamics, instead converging to local optima.\\

This work extends existing work around procedural generation of mazes using cellular automata\cite{adams2017procedural} through improved learning processes, fitness functions, and post-processing algorithms. However, this is the first work to use evolutionary algorithms to learn the full rule dynamics of 2D cellular automata. In this regard, it sets a precedent for life-like CA, showing that all tested examples can be efficiently learnt with genetic algorithms. With optimizations to the learning procedure or additional computing power, future work could validate these results on larger test sets. Although this work is not able to learn full rule dynamics for Gray-Scott systems, future work can use the simulators and toolkit developed to test more advanced evolutionary computation techniques. The software developed in this project can also be used to conduct learning experiments on any variety of discrete or continuous CA systems.\\