
Since the 1970s, John Conway's Game of Life, often shortened to just "Life", has garnered academic interest as an example of complexity emerging from simplicity without a designer. Since then, cellular automata (CA) have been used as models of physical\cite{wolf2004lattice}, chemical\cite{gray1983autocatalytic}, biological\cite{deutsch2021bio}, and social phenomena\cite{white2000high}. Recent work has used deep learning to learn CA models that converge to complex stable states (morphogenesis)\cite{mordvintsev2020growing} or recover CA transition functions from a set of observations (full rule dynamics)\cite{wulff1992learning}.\\

We investigate the use of evolutionary algorithms to this end. We build a cellular automaton simulator and evolutionary algorithm toolkit to address 3 learning objectives. The first is to elicit desirable states in life-like CA, the discrete binary-state family of models in which Life belongs. In particular, we use life-like CA to build a procedural maze generator. We learn transition functions that produce mazes with desirable qualities like long solution paths and numerous dead ends. The second objective is to learn the full rule dynamics of life-like CA. We show that genetic algorithms can learn full rule dynamics for 100\% of tested life-like CA while traversing less than 0.05\% of the search space. The third objective is to extend these methods to Gray-Scott systems, the family of continuous state automata that simulate the reactions of two diffusive chemicals. In this case, we find that evolutionary algorithms struggle to learn full rule dynamics, instead converging to local optima.\\

This work extends existing work around procedural generation of mazes using cellular automata\cite{adams2017procedural} through improved learning processes, fitness functions, and post-processing algorithms. This is the first work to use evolutionary algorithms to learn the full rule dynamics of 2D cellular automata. It sets a precedent for life-like CA, showing that all tested examples can be efficiently learnt with genetic algorithms. Although this work is not able to learn full rule dynamics for Gray-Scott systems, the simulators and toolkit developed allow for easy future testing of advanced evolutionary computation techniques. The software developed can also be used to conduct simulation experiments on any variety of discrete or continuous CA systems.\\