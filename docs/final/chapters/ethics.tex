\chapter{Ethical Considerations}

As a relatively mathematical project, there are no major ethical factors to consider here. However, there are still some legal and societal issues that are worth discussing, especially given that technologies falling in the broad field of self-organising systems, distributed systems, and automated systems can be misused.\\

The only area of legal concern is the infringement of copyright law when selecting training data. To train cellular automata for real world applications, we many use datasets collected from physical, biological, or chemical experiments. As a collection of facts, such data is exempt from copyright lawe. However, training data can, in theory, be derived works that do fall under copyright such as terrain maps or urban land use reports. When choosing point data to train on, we will ensure that they are not derived from works that fall under copyright restrictions and that they are legally suitable for academic use.\\

The only area of societal or professional concern is the general application of cellular automata in distributed technology like swarm robotics. Such systems can have military applications. It could be argued that this research could pave the way for highly distributed swarms of drones or ground robots that are capable of complex self-organising behaviour. However, this is unlikely to be a true concern since we only discuss theoretical concepts in this paper with little discussion about physical applications in robotics. Furthermore, this research is only in its very preliminary stages and the academic benefits of researching self-organising cellular automata far outweigh the negligible potential contribution that this research could have to improving malicious swarm robotics systems.\\

