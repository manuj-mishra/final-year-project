\chapter{Introduction}

\section{Overview}

Evolutionary algorithms (EAs) have long been held as effective tools in black-box optimisation problems. Grounded in the principles of Darwinian evolution, EAs traverse over a search space by performing selection, mutation, and crossover on a population of candidate solutions. Increasingly strong solutions are discovered as the fitness of the population grows.\\

Cellular automata (CAs) are discrete models of computation in which multiple "cells" are simultaneously updated at regular time steps such that the state of any cell depends exclusively on the state of the cells in a local neighbourhood around it in the previous time step. CAs are powerful computational engines due to their inherently parallel structure. However, they are more commonly utilised as an abstract representation to study nonlinear dynamics and the emergence of complexity.\\

Top-down investigations into CA behaviour are vast and varied. Mathematical analyses seek to taxonomize CA properties and prove general results about long-term behaviour from intrisic properties. In the natural and social sciences, CAs are designed to model physical, biological, or human behaviour. Both of these endeavours seek to analyse the behaviour of a CA from its structure, transition function, and possibly initial conditions.\\

In this thesis, we explore a bottom-up approach where we deduce the underlying properties of a CA by observing its behaviour. In particular, we utilise EAs to search the rulespaces of several classes of CA. We tackle multiple objectives from imitation of particular CA behaviour to generation of desirable long-term states. The ability of a rule to compactly encode the future development of a CA is reminiscent of DNA as an encoding mechanism for the development of biological entities. Noting this, it is very apt that we use biologically-inspired algorithms to optimise CAs.\\

\section{Contributions}

\begin{enumerate}
    \item We design a genetic algorithm to learn the transition rule of binary outer-totalistic (aka life-like) cellular automata.
    \item We use this to build a maze generator that utilises life-like cellular automata to procedurally generate mazes with particular properties based on user preference.
    \item We design an evolutionary strategy algorithm to learn to extrapolate and fully simulate Gray-Scott diffusion-reaction equations.
    \item We use this to build a network generator that utilises diffusion-reaction continuous automata to procedurally generate efficient networks.
\end{enumerate}