\chapter{Introduction}

\section{Overview}

Evolutionary algorithms (EAs) have long been held as effective tools for black-box optimisation problems. Grounded in the principles of Darwinian evolution, EAs traverse over a search space by performing selection, mutation, and crossover on a population of candidate solutions. Increasingly strong solutions are discovered as the fitness of the population grows.\\

Cellular automata (CAs) are discrete models of computation in which multiple "cells" are simultaneously updated at regular time steps such that the state of any cell depends exclusively on the state of the cells in a local neighbourhood around it in the previous time step. CAs are powerful computational engines due to their inherently parallel structure. However, they are more commonly utilised as an abstract representation to study nonlinear dynamics and the emergence of complexity.\\

Top-down investigations into CA behaviour are vast and varied. Mathematical analyses seek to taxonomize CA properties and prove general results about long-term behaviour from intrisic properties. In the natural and social sciences, CAs are designed to model physical, biological, or human behaviour. Both of these endeavours seek to analyse the behaviour of a CA from its structure, transition function, and possibly initial conditions.\\

In this thesis, we explore a bottom-up approach where we deduce the underlying properties of a CA by observing its behaviour. In particular, we utilise EAs to search the rulespaces of several classes of CA. We tackle multiple optimisation problems from the imitation of particular CA behaviour to generation of desirable long-term states.\\

\section{Objectives}

We aim to develop a system to discover cellular automata that are highly likely to exhibit a desired behaviour. 

\section{Contributions}
The key contributions of this project are as follows:
\begin{enumerate}
    \item \textbf{Evolutionary Algorithm Toolkit}\\ A versatile toolkit that implements multiple evolutionary algorithms to train and optimise different classes of CA. We use this to successfully predict the update rule of binary outer-totalistic CA from observations of the CA running on random initial conditions. We show that this can be extended to continous automata by predicting the parameters of diffusion-reaction equations from simulations of chemical reactions.
    \item \textbf{Cellular Automaton Simulator}\\ A system that can efficiently simulate discrete and continuous cellular automata. This allows a broad range of fitness functions to be implemented in the EA toolkit. This can also render snapshots of the CA directly during simulation which allows animations to be efficiently generated afterwards.
    \item \textbf{Procedural Maze Generator}\\ A CA-based maze generation program that uses the EA toolkit to produce difficult mazes with characteristics optimised to user preference.
    
\section{Technical Challenges}

\end{enumerate}